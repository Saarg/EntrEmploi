\documentclass[a4paper, 12pt]{article}
\usepackage[utf8]{inputenc}
\usepackage{graphicx}
\usepackage{xcolor}
\usepackage{appendix}
\usepackage[top=2cm, bottom=2cm, left=2cm, right=2cm]{geometry}
\usepackage[francais]{babel}
\usepackage{hyperref}
\hypersetup{
    colorlinks=true,
    linkcolor=black,
    urlcolor=black,
    linktoc=all
}

\usepackage{fancyhdr}
\usepackage{lastpage}

\addtocontents{toc}{\protect\thispagestyle{fancy}}

\pagestyle{fancy}
\setlength\headheight{12mm}
\fancyhf{}
\lhead{\includegraphics[height=1cm]{../../public/images/Logo-spf-sm.png}}
\chead{}
\rhead{\includegraphics[height=8mm]{../../public/images/Logo-entremploi-sm.png}}
\lfoot{\today}
\cfoot{}
\rfoot{Fiche synthèse}

\usepackage{multicol}

\begin{document}
\begin{center}\Huge Site Web Entr'Emploi\end{center}
\vskip2cm

\begin{multicols}{2}
\section*{Entr'Emploi}
Entr'Emploi propose un accompagnement personnalisé pour les chercheurs d'emploi plus ou moins éloigné du marché du travail depuis cette année dans la région de Brest.Ils les aident à clarifier leur projet, s'orienter vers des formations et à organiser leur recherche d'emploi.

\section*{Plan du Site}
Le plan du site représente le squelette de notre travail, il liste les fonctionnalités principales du site que nous souhaitons mettre en place.
\begin{itemize}
    \item Présentation du projet
    \item Prestations (aide au CV, simulation d'entretien...)
    \item CVthèque
    \item Offres d'emploi
    \item Liste de partenaires
    \item Lien vers la fédération/don/bénévolat
\end{itemize}

\section*{Notre projet}
La création du site doit permettre a l'équipe du secours populaire de se faire connaître et d'aider les demandeurs d'emploi à trouver une panoplie d'offres d'emplois et surtout de faire
appel aux services proposés par Entr'Emploi (Ateliers CV, estime de soi, simulation d'entretien).\\
Le site doit également servir de vitrine au projet Entr'Emploi dans le but d'étendre l’initiative a d'autres villes.

\section*{Secours Populaire}
%\includegraphics[width=6cm]{../../public/images/Logo-spf.png}
Nous sommes déjà en contact avec l'équipe de Brest que nous avons rencontré a plusieurs reprise. Nous avons Définis le plan du site ci-contre avec eux pour répondre le mieux possible à leur besoin.\\
Le budget nécessaire a l’hébergement du site est pris en charge par le secours populaire
\end{multicols}

\begin{center}
\vskip2cm
\begin{tabular}{p{4cm} p{4cm} p{4cm}}
\begin{center}Maillard Noé\end{center} 
&
\begin{center}Parain: Mr Redou\end{center}
&
\begin{center}Milsonneau Jean\end{center}
\end{tabular}
\end{center}

\includegraphics[width=16cm]{banner.png}

\end{document}